Príbehy nás sprevádzajú celým životom, či už sú to naše vlastné, našich príbuzných, kamarátov alebo ľudí ktorých ani nepoznáme. Od malička ich čítame, stretávame sa s nim ai vyhľadávame ich. Z vlastnej skúsenosti viem, že človeka dokážu pohltiť, primeť ho k zamysľeniu alebo jednoducho si pri nich človek oddýchne. Čo ich ale robí tak zaujímavými?\par
Už starovekí gréci sa zaoberali písaním a štúdiom tvorby príbehov. Z naratológie, ktorej históriu datujeme až ku starovekým grékom vieme, že príbeh by mal mať nejakú dejovú štruktúru aby nás zaujal a príbeh by nemal mať veľa hlavných postáv, potom by sa mohlo stať že sa čitateľ v príbehu stratí. Navyše tvorba príbehov je komplexný proces, pri ktorom my ludia veľa vecí robíme tak nejak automaticky.\par
Zaujímavá problematika nastáva, keď túto tvorbu príbehov dáme za úlohu počítaču. Keďže problém je komplexný a navyše tažko definovateľný (veľa procesov robíme v hlave automaticky), vzniká pre programátora netriviálna úloha ako sa s týmto problémom popasovať. Takýmito úlohami sa zaoberá vedná disciplnína výpočtová naratológia, ktorá študuje tvorbu príbehov z výpočtového hľadiska. Zameriava sa na algoritmické procesy ktoré vytváraju a interpretujú príbehy a tiež na modelovanie štruktúry príbehu z hľadiska vypočítateľných reprezentácií.\par
Počítačové hry sú mojou záľubou, ktorá ma sprevádza už odmalička a teda mi bolo prirodzené zasadiť problematiku tvorby príbehov do počítačovej hry. Hra príbeh zinterktívni a autori príbehu majú k dispozícii viacero nástrojov ako hráča viac vtiahnuť do deja. Môžu použiť rôzne obrázky, zvuky, animácie, ktoré príbeh dotvárajú a vytvárajú určitú atmosféru. Kombináciou tvorby príbehov a tvorby hier si myslím že vzniká zaujímavá téma na študovanie.\par
Cieľom tejto práce je teda vytvoriť systém, ktorého súčasťou bude jednoduchá textová hra, ktorý bude generovať zaujímavý príbeh. Tento príbeh sa zakaždým vygeneruje nanovo, teda hru bude možné hrať znovu a znovu vždy s iným príbehom. Keďže ide o hru tak posun v príbehu bude záležať od interakcie hráča, ktorý hru hrá. Ďaľším aspektom je zaujímavosť príbehu, to znamená, že budeme skúmať dynamiku daného príbehu, čiže či sa nebudú často opakovať tie isté akcie, či akcie budú na seba logicky nadväzovať alebo či budú zmysluplné z hľadiska deju atď.\par
V prvej kapitole práce si popíšeme základné pojmy týkajúce sa našej práce. Spomenieme tiež základy výpočtovej naratológie a jej systémy, ktoré sa používajú na generovanie príbehov. Budú tu popísané aj metódy, ktoré využijeme v našej práci.\par
Druhou kapitolou je špecifikácia našeho systému, kde popíšeme čo bude naša aplikácia spĺňať.\par
Potom v tretej kapitole navrhneme riešenie špecifikácie, teda akým spôsobom budeme riešiť úlohy opísané v špecifikácii.\par
Štvrtou kapitolou bude implementácia, ktorá bude pojenávať o konkrétnych riešeniach navrhnutých úloh. Spomenieme tu aj rozdiely, ktoré nastali oproti návrhu systému.\par
Na koniec v piatej kapitole demoštruujeme výsledky našej aplikácie a ukážeme si aj nejaké štatistiky spojené s generovaním príbehov.


