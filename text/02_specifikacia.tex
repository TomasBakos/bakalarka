V nasledovnej kapitole si popíšeme, čo by mala naša aplikácia robiť aby spĺňala cieľ zadania našej bakalárskej práce. Z cieľa našej práce vyplýva, že chceme vytvoriť systém, ktorý bude vytvárať zaujímavé a interaktívne príbehy. Tieto príbehy následne budú hratelné v našej dobrodružnej textovej hre.\par
\section{Dobrodružná textová hra}
Dobrodružná textová hra je dobrý nástroj ako zinteraktívniť nejaký príbeh. Používateľ sa pohybuje vo svete vlastným tempom a interaguje s daným vymysleným svetom podľa vlastného uváženia. Naša aplikácia bude implementovať aj jednoduchú dobrodružnú textovú hru. Bude teda bude ponúkať používateľovi možnosti ako prechádzať daným herným svetom. Keďže ide o textovú hru tak, všetká komunikácia bude v textovej forme. Opis sveta a ostatných herných prvkov ako napríklad inventár hráča a aj ovládanie sa bude vykonávať cez konzolu. Ovládanie našej aplikácie bude formou príkazov do konzoli. Tieto príkazy budú viditeľné hráčovi na konzole a hráč si bude môcť vybrať, ktorý chce použit.\par
\section{Herné objekty}
Herné objekty, ktoré budú použité v hernom svete, budú popísané v preddefinovaných súboroch. Na začiatku sa tieto súbory spracujú a hodnoty sa zaznamenajú do našej aplikácie. Z týchto hodnôt sa následne vygenerujú všetky možné akcie, ktoré budú neskôr tvoriť príbeh hry. Súbory by mali byť ľahko upravovateľné. Toto zaručí jednoduché modifikovanie herného sveta a príbehov v našej hre.\par
\section{Generovanie sveta a príbehu}
Herný svet bude náhodne generovaný podľa učitých pravidiel. Pravidlá budú definované zápletkami, ktoré sa do našej hry vymyslia. Použité náhodné generovanie sveta zaistí znovuhrateľnosť príbehov, teda že príbeh bude zakaždým rozdielny. Po náhodnom vygenerovaní sveta sa následne naplánuje príbeh v rámci tohto sveta. Naplánovaný príbeh zaistí hrateľnosť vygenerovaného sveta. Príbeh vlastne zistí, či je daný svet vhodným kandidátom do našej hry. Keďže sa v danom svete našiel príbeh, tak vieme, že sa tento svet dá dohrať v našej hre. Následne sa ohodnotí zaujímavosť naplánovaného príbehu. Toto sa bude diať skúmaním štruktúry a rôznych vlastností naplánovaného príbehu. Ako napríklad, či sa neopakuje nejaká akcia v príbehu veľakrát za sebou. Po vygenerovaní určitého počtu svetov sa zistí, ktorý svet má najlepšie hodnotenie jeho príbehu. Po tomto zistení sa hráčovi ponúkne najlepší vygenerovaný príbeh a môže sa ho zahrať.\par
\section{Generovanie do súboru}
Ďaľšou funkcionalitou našej aplikácie bude možnosť generovať tieto svety do súboru. Súbor by mal byť vo formáte, ktorý je ľahko čitateľný človeku. Používateľ si bude môcť vybrať počet náhodne generovaných svetov. Svety sa budú generovať do preddefinovaného priečinka a používateľ si následne bude môcť prezerať tieto vygenerované svety.\par
\section{Generovanie zo súboru}
Treťou hlavnou funkcionalitou aplikácie bude schopnosť nášho programu načítať používateľom zadaný súbor vo formáte, ktorý bude použitý pri generovaní do súboru. Tento súbor sa spracuje a vytvorí sa reprezentácia sveta podľa tohto súboru. Následne sa náš program pokúsi naplánovať príbeh nad načítaným svetom. Ak sa príbeh podaril naplánovať tak ho nechá hrať používateľovi, ak nie tak program upozorní používateľa, že daný svet je nevhodný. Takto si používateľ môže jednoducho vytvoriť vlastný svet a skúsiť si ho zahrať.\par
\section{Nastavenie generátora hodnôt}
Náhodné generovanie sveta sa bude generovať podľa pseudo-náhodného generátora hodnôt. Tieto generátory používajú takzvané semiačka na generovanie hodnôt. Používateľ bude mocť pri generovaní sveta v našej aplikácii zadať hodnotu semiačka aby sa mu podľa ňej vygeneroval svet. Bude to možné pri generovaní do súboru aj pri generovaní sveta rovno na hranie. Taktiež sa hodnota semiačka bude zobrazovať pri náhodnom generovaní sveta pred začiatkom hry. Týmto spôsobom si potom používateľ može znovu zahrať nejaký príbeh, ktorý už hral. A môže si ho aj nechať vygenerovať do súboru a prezrieť si daný svet.