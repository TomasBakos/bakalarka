Pribehy nas sprevadzaju celym zivotom, ci ich uz su to nase vlastne, nasich pribuznych, kamaratov alebo ludi ktorych ani nepozname. Od malicka ich citame a stretavame sa s nimi vyhľadávame ich. Z vlastnej skúsenosti viem, že človeka dokážu pohltiť, primeť ho k zamysľeniu alebo jednoducho si pri nich človek oddýchne. Čo ich ale robí tak zaujímavými? Už starovekí gréci sa zaoberali písaním a štúdiom tvorby príbehov. Z naratologie, ktorej históriu datujeme až ku starovekým grékom vieme, že príbeh by mal mať nejakú dejovú štruktúru aby nás zaujal a príbeh by nemal mat veľa hlavných postáv, potom by sa mohlo stať že sa čitateľ v príbehu stratí. Navyše tvorba príbehov je komplexný proces, pri ktorom my ludia veľa vecí robíme tak nejak automaticky.\par
Zaujímavá problematika nastáva, keď túto tvorbu príbehov dáme za úlohu počítaču. Keďže problém je komplexný a navyše tažko definovatelný (veľa procesov robíme v hlave automaticky), vzniká pre programátora netriviálna úloha ako sa s týmto problémom popasovať. Takýmito úlohami sa zaoberá vedná disciplnína výpočtová naratológia, ktorá študuje tvorbu príbehov z výpočtového hľadiska. Zameriava sa na algoritmické procesy ktoré vytváraju a interpretujú príbehy a tiež na modelovanie štruktúry príbehu z hladiska vypočítateľných reprezentácií. Povieme si o nej viac v ďaľšej kapitole.\par
Počítačové hry sú mojou záľubou, ktorá ma sprevádza už odmalička a teda mi bolo prirodzené zasadiť problematiku tvorby príbehov do počítačovej hry. Hra príbeh zinterktívni a autori, programatori príbehu majú k dispozícii viacero nástrojov ako hráča viac vtiahnuť do deja. Môžu použiť rôzne obrázky, zvuky, animácie, ktoré príbeh dotvárajú a vytvárajú určitú atmosféru. Kombináciou tvorby príbehov a tvorby hier si myslím že vzniká zaujímavá téma na študovanie.\par
Cieľom tejto práce je teda vytvoriť jednoduchú textovú hru, ktorá bude generovať dobre rozvinutý príbeh. Tento príbeh sa zakaždým vygeneruje nanovo, teda hru bude možné hrať znovu a znovu vždy s iným príbehom. Keďže ide o hru tak posun v príbehu bude záležať od interakcie hráča ktorý hru hrá. Ďaľším aspektom je, že príbeh buďe "zaujímavý", to znamená, že sa nebudú často opakovať tie isté akcie, akcie budú na seba logicky nadväzovať, budú zmysluplné z hľadiska deju atď. Spôsob akým sa dajú vytvárať takéto príbehy popíšeme v nasledujúcej kapitole.


